%\documentclass[varwidth, border={0 0 0 0}]{standalone}
%\documentclass[border={0 0 0 0}]{standalone}
\documentclass[10pt,a4paper]{standalone}
\usepackage[usenames,dvispnames,svgnames,table]{xcolor}
\usepackage{multirow}
\usepackage{helvet}
\usepackage{amsmath}
\usepackage{rotating}
\usepackage{graphicx}
\renewcommand{\familydefault}{\sfdefault}
\usepackage{setspace}
\usepackage{caption}
\captionsetup{labelformat=empty}
\usepackage{tikz}
\usetikzlibrary{positioning,shapes,arrows,calc,fit,decorations.pathmorphing,shapes.geometric}
\usepackage{centernot}
\setlength{\tabcolsep}{2pt}

\begin{document}
\tikzset{ module/.style={draw, rectangle},label/.style={ } }
\tikzstyle{block1} = [rectangle, draw, text width=5em, text centered, rounded corners, minimum height=4em]
\tikzstyle{block2} = [rectangle, draw, text width=8em, text centered, rounded corners, minimum height=4em]

%\begin{tikzpicture}[node distance = 2cm, auto]
\begin{tikzpicture}[>=latex]
%\begin{tikzpicture}

\node [rectangle,draw] (1E) {Assessed for eligibility (n=600)};
\node [rectangle,draw, below right =10pt and -50 pt of 1E, align=left] (Ex) {Excluded (n=2) \\
  \begin{tabular}{>{\bfseries}r l}
    2 & Ineligble \\
    0 & Declined \\
    0 & Other
  \end{tabular}
  };
\node [rectangle,draw, below =80 pt of 1E] (1R) {Randomized (n=598)};

%%%%%%%%%%%%%%%%%%%%%%%%%%%%%%%%%%%%%%%%%%%%%
% ALLOCATION
%%%%%%%%%%%%%%%%%%%%%%%%%%%%%%%%%%%%%%%%%%%%%
\node [rectangle,draw,below left = 30pt and -15 pt of 1R, align=left] (1I) {Allocated to intervention (n=298) \\
  \begin{tabular}{>{\bfseries}r l}
    188 & Received intervention \\
    95  & Did not receive intervention \\
    & \begin{tabular}{>{\bfseries}r l}
        X & Did not receive reason 1 ... \\
        Y & Did not receive reason 2 ...
      \end{tabular}
  \end{tabular}};
\node [rectangle,draw,below right =30pt and -15 pt of 1R, align=left] (1P) {Allocated to placebo (n=300) \\
  \begin{tabular}{>{\bfseries}r l}
    0 & Received placebo \\
    0 & Did not receive placebo \\
    & \begin{tabular}{>{\bfseries}r l}
        X & Did not receive reason 1 ... \\
        Y & Did not receive reason 2 ...
      \end{tabular}
  \end{tabular}
  };
\node (X) at ($(1I)!0.5!(1P)$) [gray, align=center] {Allocation};

%%%%%%%%%%%%%%%%%%%%%%%%%%%%%%%%%%%%%%%%%%%%
% FOLLOW-UP
%%%%%%%%%%%%%%%%%%%%%%%%%%%%%%%%%%%%%%%%%%%%
\node [rectangle,draw, below = of 1I, align=left] (IF) {
Lost to follow-up (n=XX) \\
  \begin{tabular}{>{\bfseries}r l}
    X & Lost reason 1 ... \\
    Y & Lost reason 2 ...
  \end{tabular} \\
Discontinued intervention (n=XX) \\
  \begin{tabular}{>{\bfseries}r l}
    X & Discontinued reason 1 ... \\
    Y & Discontinued reason 2 ...
  \end{tabular}
  };
\node [rectangle,draw, below = of 1P, align=left] (PF) {
Lost to follow-up (n=XX) \\
  \begin{tabular}{>{\bfseries}r l}
    X & Lost reason 1 ... \\
    Y & Lost reason 2 ...
  \end{tabular} \\
Discontinued intervention (n=XX) \\
  \begin{tabular}{>{\bfseries}r l}
    X & Discontinued reason 1 ... \\
    Y & Discontinued reason 2 ...
  \end{tabular}
  };
\node (Y) at ($(IF)!0.5!(PF)$) [gray, align=center] {Follow-Up};

%%%%%%%%%%%%%%%%%%%%%%%%%%%%%%%%%%%%%%%%%%%%%%
% ANALYSIS
%%%%%%%%%%%%%%%%%%%%%%%%%%%%%%%%%%%%%%%%%%%%%%

\node [rectangle,draw, below = of IF, align=left] (IA) {
Analysed (n=XX) \\
  \begin{tabular}{>{\bfseries}r l}
    X & Excluded from analysis\\
    & \begin{tabular}{>{\bfseries}r l}
        X & Exclusion reason 1 ... \\
        Y & Exclusion reason 2 ...
      \end{tabular}
  \end{tabular}
  };
\node [rectangle,draw, below = of PF, align=left] (PA) {
Analysed (n=XX) \\
  \begin{tabular}{>{\bfseries}r l}
    X & Excluded from analysis\\
    & \begin{tabular}{>{\bfseries}r l}
        X & Exclusion reason 1 ... \\
        Y & Exclusion reason 2 ...
      \end{tabular}
  \end{tabular}
  };
\node (Z) at ($(IA)!0.5!(PA)$) [gray, align=center] {Analysis};

%%%%%%%%%%%%%%%%%%%%%%%%%%%%%%%%%%%%%%%%%%%%%%%
% PATHS
%%%%%%%%%%%%%%%%%%%%%%%%%%%%%%%%%%%%%%%%%%%%%%%

\path[->] (1E) edge (1R);
\path[->] (1R) edge (1I);
\path[->] (1R) edge (1P);
\path[->] (1I) edge (IF);
\path[->] (1P) edge (PF);
\path[->] (IF) edge (IA);
\path[->] (PF) edge (PA);
 \draw[ ->](1E) |- (Ex);

\end{tikzpicture}

\end{document}
